\documentclass[9pt]{extarticle}
\usepackage[a4paper,headsep=10pt, top=30pt, bottom=30pt, left=30pt, right=30pt, footskip=10pt]{geometry}
\usepackage{listings}
\usepackage{setspace}			%used for \begin{spacing}\end{spacing}-> vertical space setting
\usepackage{amsfonts} 		%used for \mathbb{} -> numerical sets
\usepackage{amsmath}			%used for \text{} inside math equations
\usepackage{enumitem}
\usepackage{bm}					%used for bold math expressions
\usepackage{calc}				%used for algebric operations inside \newenvironment
\usepackage{graphicx}			%used for custom limits
\usepackage{fancyhdr}			%used for head and foot settings
%\usepackage{draftwatermark}		%used for watermark
%
%
%%%watermark settings
%\SetWatermarkText{Vici Francesco}
%\SetWatermarkScale{2}
%\SetWatermarkAngle{60}
%%end watermark settings



\fancypagestyle{plain}
{
\fancyhead{}\fancyfoot{}
\fancyhead[C]{\vspace{0pt}   Formulario Analisi II}
\fancyfoot[R]{Vici Francesco}
\fancyfoot[C]{\thepage}
}
\pagestyle{plain}



\newenvironment{formulario}
{
\setlength{\columnsep}{3em}
\twocolumn
\lstset{tabsize=3}
\begin{spacing}{1}
\begin{flushleft}
}{
\end{flushleft}
\end{spacing}
}



\newenvironment{tcenter}{
  \par
  \centering
  \setlength{\parskip}{0pt} % Rimuovi spaziatura verticale
  \noindent
}{
  \par
}



\newenvironment{descr}[1]
{
\setlist[description,1]{leftmargin=2em + (2em * #1),labelindent=0em + (2em * #1)}
\begin{description}[topsep=0pt,itemsep=0pt,partopsep=0pt, parsep=0pt]
}{
\end{description}
}



\newenvironment{myParagraph}[1]
{
\begin{tcenter}
\textbf{#1}
\end{tcenter}
}{
\myRule
}



\newcommand{\norma}[1]{\lvert\lvert #1\rvert\rvert}
\newcommand{\module}[1]{\lvert #1\rvert}
\newcommand{\R}{\mathbb{R}}
\newcommand{\N}{\mathbb{N}}
\newcommand{\myRule}{\rule{250pt}{0.1pt}}
\newcommand{\Lim}[1]{\raisebox{0.5ex}{\scalebox{0.8}{$\displaystyle \lim_{#1}\;$}}}

\newcommand{\scalare}{\mathbin{\vcenter{\hbox{\scalebox{.6}{\;$\bullet$\;}}}}}



\begin{document}

	\begin{formulario}

%PARAGRAFO 1
		\begin{tcenter}
\textbf{FUNZIONI A PIÙ VARIABILI}
		\end{tcenter}
Si dice funzione a più variabili una funzione del tipo: 
		\begin{tcenter}
$f:\R^n\to\R^m$
		\end{tcenter}
Inoltre possiamo definire "scalare" una funzione del tipo $f:\R^n\to\R$ mentre viene definita vettoriale una funzione del tipo $f:\R^n\to\R^m$.
\myRule

%PARAGRAFO 2
		\begin{tcenter}
\textbf{FUNZIONI A DUE VARIABILI SCALARI}
		\end{tcenter}
È una funzione del tipo $f:\R^2\to\R$, chiamo le variabili $x,y$. \\
		\begin{descr}{0}
\item[Dominio] $\rightarrow$ In generale, dati $f:D\to\R$ con $D\subset\R^2$, $D$ viene detto dominio.
\item[Segno] $\rightarrow$ Studio dove $f:(x,y)>0,\text{ }=0 \text{ oppure }<0$. Si tratta quindi di determinare uno di questi sottoinsiemi (spesso è utile rappresentarlo graficamente). 
		\end{descr}
Al fine di definire la continuità di una funzione è necessario definire i concetti di "distanza" e tutti i concetti che ne derivano e quello di "limite":
		
		\begin{descr}{1}
\item[Distanza in $\R^2$] $\rightarrow$ Dati due punti $(x,y) \text{ e } (x_0,y_0)$ la distanza tra loro è data dalla norma a 2, cioè: \\
			\begin{tcenter}
$\mathbf{\norma{(x,y)-(x_0,y_0)}=\sqrt{(x-x_0)^2+(y-y_0)^2}}$
			\end{tcenter}
\item[Intorno (Circolare)] $\rightarrow$ Dato un punto $(x_0,y_0)$, $I$ viene detto intorno circolare di $(x_0,y_0)$ se:\\
			\begin{tcenter}
$\mathbf{I(x_0,y_0)=\{(x,y)\;\big\vert\;\norma{(x-y)-(x_0,y_0)}<r\}}$
			\end{tcenter}
\item[Interno] $\rightarrow$ Dato un punto $(x_0,y_0)\in A$, tale punto si dice interno ad $A$ se:\\
			\begin{tcenter}
$\mathbf{\exists  I(x_0,y_0)\subset A}$
			\end{tcenter}
\item[Frontiera (o Bordo)] $\rightarrow$ Un punto $(x_0,y_0)$ si dice appartenere alla frontiera (o bordo) di $A$, e si scrive $(x_0,y_0)\in\partial A$ se:\\
			\begin{tcenter}
$\mathbf{\forall I(x_0,y_0)
			\begin{cases}
   	\exists (x,y)\neq(x_0,y_0)\text{ t.c. } (x,y)\in A\cap I(x_0,y_0) \\
   	\exists (z,t)\neq(x_0,y_0)\text{ t.c. } (z,t)\in A^c\cap I(x_0,y_0)
  			\end{cases}}$
			\end{tcenter}
\item[Punto di Accumulazione] $\rightarrow$ Dato un punto $(x_0,y_0)$, questo si dice punto di accumulazione per $A$ se: \\
			\begin{tcenter}
$\forall I(x_0,y_0)\; \exists(x.y)\neq(x_0,y_0) \text{ t.c. } (x,y)\in I(x_0,y_0)\cap A$
			\end{tcenter}
\item[Insieme Aperto/Chiuso] $\rightarrow$ A è un insieme aperto se coincide con l'insieme dei sui punti interni (cioè $\mathbf{A}$ \textbf{ha tutto il bordo "tratteggiato"}), mentre A è un insieme se $\partial A\subset A$ (cioè $\mathbf{A}$ \textbf{ha tutto il bordo "continuo"}).

\item[Chiusura di un Insieme] $\rightarrow$ La chiusura di un insieme $A$ è data dall'insieme stesso unito con la sua frontiera, cioè:
			\begin{tcenter}
		$\mathbf{\overline{A}=A\cup\partial A}$
			\end{tcenter}
		\end{descr}
Inoltre possono anche essere utili altre definizioni secondarie come:
		\begin{descr}{1}
\item[Insieme Limitato] $\rightarrow$ Un insieme $A$ è limitato in $\R^2$ se: 
			\begin{tcenter}
$\mathbf{\exists a,b,c,d \in \R \text{ t.c. } \forall (x,y)\in A: a\leq x\leq b,\; c\leq y\leq d}$\\ 
oppure \\
$\mathbf{\exists c\in\R^+ \text{ t.c. } \forall (x,y)\in A: x^2+y^2\leq c }$  
			\end{tcenter} 
\item[Insieme non Limitato] $\rightarrow$ Un insieme $A$ è non limitato se:
			\begin{tcenter}
$\mathbf{\forall c\; \exists (x,y) \text{ t.c. } \norma{(x,y)}>c}$ 
			\end{tcenter}
\item[Insieme Compatto] $\rightarrow$ Un insieme si dice compatto se è chiuso e limitato.
\item[Insieme Connesso] $\rightarrow$ Un insieme si dice connesso non ai può ricoprire con due insiemi non vuoti, aperti e disgiunti.
		\end{descr} 
\myRule

%PARAGRAFO 3
		\begin{tcenter}
\textbf{LIMITI DI FUNZIONI IN DUE VARIABILI}
		\end{tcenter}
La principale differenza con i limiti di funzioni in una variabile sta nel fatto che in $\R$ esistono solo due possibili "direzioni" per cui una funzione può tendere ad un punto, mentre in $\R^2$ le direzioni sono infinite.
		\begin{descr}{0}
\item [\underline{DEF:} (Per Intorni)] $\rightarrow$ Data $f:D\to\R,D\subset\R^2$ e dato $(x_0,y_0)$ punto di accumulazione per $D$ dico che $\Lim{(x,y)\to(x_0,y_0)}f(x,y)=l$ se $\forall\epsilon >0 \; \exists\delta>0 \;$ t.c. se $(x,y)\in D, \; (x,y)\neq(x_0,y_0)$:
			\begin{tcenter}
$\mathbf{\norma{(x,y)-(x_0,y_0)}<\delta \implies \module{f(x,y)-l}<\epsilon}$
			\end{tcenter}
\item [\underline{DEF:} (Per Successioni)] $\rightarrow$ Data $f:D\to\R,D\subset\R^2$ e dato $(x_0,y_0)$ punto di accumulazione per $D$ dico che $\Lim{(x,y)\to(x_0,y_0)}f(x,y)=l$ se $\forall (x_n,y_n)\in D\setminus{(x_0,y_0)}, n\in \N$ t.c. se:
			\begin{tcenter}
$\mathbf{(x_n,y_n)\to (x_0,y_0) \text{ allora } f(x_n,y_n)\to l}$
			\end{tcenter}
		\end{descr}
È anche possibile approcciare il problema sotto il punto di vista delle coordinate polari $(\rho,\Theta)$, questo riduce il problema ad un limite ad una variabile in quanto il parametro $\Theta$ è libero è dobbiamo verificare solo la tendenza di $\rho$.
\myRule

%PARAGRAFO 4
		\begin{tcenter}
\textbf{CONTINUITÀ DI FUNZIONI A DUE VARIABILI}
		\end{tcenter}
		\begin{descr}{0}
\item[\underline{DEF:}] $\rightarrow$ Data $f:D\to\R, D\subset\R$ e dato $(x_0,y_0)\in D$ punto di accumulazione per D, si dice che $f$ è continua in $(x_0,y_0)$ se:
			\begin{tcenter}
$\mathbf{\exists\Lim{(x,y)\to(x_0,y_0)}f(x,y) \text{ \textbf{ed è pari a} } f(x_0,y_0)}$.
			\end{tcenter}
		\end{descr}
Se la funzione è definita "a tratti" allora la continuità è da verificare nei punti di frontiera comune.
		\begin{descr}{0}
\item[Teorema di Weierstrass] $\rightarrow$ Data $f:K\to\R$ con $K$ compatto e $f$ continua in $\R$ allora:
			\begin{tcenter}
$\mathbf{f}$ \textbf{ha massimo e minimo assoluti in} $\mathbf{K}$
			\end{tcenter}
		\end{descr}
\myRule

%PARAGRAFO 5
		\begin{tcenter}
\textbf{RAPPRESENTAZIONE GRAFICA DI FUNZIONI A DUE VARIABILI}
		\end{tcenter}
		\begin{descr}{0}
\item[Superficie Associata] $\rightarrow$ Data $f:D\to\R,\; D\subset\R^2$ definisco la superficie associata ad $f$ come:
 			\begin{tcenter}
$\mathbf{S_f=\left\{ (x,y,z)\in\R^3 \bigg\vert 
\begin{array}{l}
	\mathbf{(x,y)\in D }\\
	\mathbf{z=f(x,y)}
\end{array}
\right\}} \mathbf{\subset\R^3}$
			\end{tcenter}
\item[Insiemi di Livello] $\rightarrow$ Data $f:D\to\R,\; D\subset\R^2$ e dato $c\in\R$, chiamo "insieme di livello di $f$ a livello $c$" l'insieme:
			\begin{tcenter}
$\mathbf{U_c:= \left\{ (x,y)\in D \big\vert f(x,y)=c \right\}\subset D}$
			\end{tcenter}
\underline{OSS:} 
			\begin{tcenter}
$U_c$ può essere: 
	$\begin{cases}
\emptyset \\
\text{insieme di punti isolati} \\
\text{curva} \\
\text{regione bidimensionale}
	\end{cases}$
			\end{tcenter}
		\end{descr}
\myRule

%PARAGRAFO 6
		\begin{tcenter}
\textbf{DERIVATE PARZIALI}
		\end{tcenter}
		\begin{descr}{0}
\item[\underline{DEF:}] $\rightarrow$ Sia $(x_0,y_0)$ un punto interno al dominio di $f$. si chiama derivata parziale di $f$ rispetto ad $x$ nel punto $(x_0,y_0)$ , la derivata classica di:
			\begin{tcenter}
$x \to f(x,y_0) \text{ in } x=x_0$
			\end{tcenter}
e si indica con $\frac{\partial f}{\partial x}(x_0,y_0)$. Ricordando la definizione di derivata classica si ha che:		
			\begin{tcenter}
$\mathbf{\frac{\partial f}{\partial x}(x_0,y_0)=\Lim{h\to 0} \frac{f(x_0+h,y_0)-f(x_0,y_0)}{h}}$
			\end{tcenter}
\underline{\textbf{OSS:}} Si possono usare le stesse regole di derivazione classiche considerando l'altra variabile come una costante. 
		\end{descr}
Le derivate parziali si mettono in un vettore chiamato gradiente. Questo, come funzione di $x,y$ è un campo vettoriale:
		\begin{tcenter}
$\nabla f(x,y)=(\frac{\partial f}{\partial x}(x,y)\; ,\; \frac{\partial f}{\partial y}(x,y))$
		\end{tcenter} 
		\begin{descr}{0}
\item[Derivate Parziali di Ordine Superiore] $\rightarrow$ Per una funzione a due variabili esistono 4 derivate di secondo ordine, infatti ogni derivata prima può essere derivata rispetto ad $x$ o rispetto ad $y$:
			\begin{tcenter}
$D^2\, f(x,y)= 
\begin{pmatrix}
\frac{\partial^2 f}{\partial^2 x} & \frac{\partial^2 f}{\partial x\partial y} \\
\frac{\partial^2 f}{\partial y\partial x} & \frac{\partial^2 f}{\partial^2 y}
\end{pmatrix}$
			\end{tcenter}
Tale matrice viene detta Matrice Essiana.
			\begin{descr}{0}
\item[Teorema di Schwarz] $\rightarrow$ Se f è di classe $C^2$, cioè le derivate seconde esistono e sono continue allora:
				\begin{tcenter}
$\frac{\partial^2 f}{\partial x\partial y}(x,y)=\frac{\partial^2 f}{\partial y\partial x}(x,y)$
				\end{tcenter}
cioè la matrice essiana è simmetrica.
			\end{descr}
		\end{descr}
		\begin{descr}{0}
\item[Differenziabilità] $\rightarrow$ Data $f:A\to\R$ con $A$ aperto di $\R^2$, $\nu\in\R^2$ ($\nu\to0$ significa $\norma{\nu}\to0$), allora $f$ si dice differenziabile  se:
			\begin{tcenter}
$\Lim{\nu \to \underline{0}} \frac{f((x,y)+\nu)-f(x,y)-\nabla f(x,y)\cdot\nu}{\norma{\nu}}=0$ 
			\end{tcenter}
			\begin{descr}{0}
\item[Teorema del Differenziale] $\rightarrow$ Data $f:A\to\R$ con $A$ aperto di $\R^2$ se:
				\begin{tcenter}
$f$ è di classe $C^1 \implies f$ è differenziabile. 
				\end{tcenter}
				\begin{descr}{0}
\item[\underline{OSS:}] Se $f$ è di classe $C^1 \implies f$ differenziabile $\implies$ $f$ è di classe $C^0 \implies f$ continua.  
				\end{descr}
			\end{descr}
		\end{descr}
\myRule

%PARAGRAFO 7
		\begin{myParagraph}{PIANO TANGENTE}
Se una funzione è differenziabile, si può approssimare "bene" con un piano. Si chiama Piano Tangente il piano in $\R^3$ di equazione:
			\begin{tcenter}
$z:=f(x_0,y_0)+\nabla f(x_o,y_0)\scalare (x-x_0,y-y_0)$
			\end{tcenter}
è della forma $z=ax+by+c$ e il punto $(x_0,y_0,f(x_0,y_0))$ è il punto di contatto tra il piano e la funzione.
		\end{myParagraph}

%PARAGRAFO 8
		\begin{tcenter}
\textbf{DERIVATA DIREZIONALE}
		\end{tcenter}
		
Dato $\overrightarrow{\nu}$ un vettore unitario di $\R^2$ (cioè $\norma{\overrightarrow{\nu}}=1$) la derivata direzionale di $f$ in direzione $\overrightarrow{\nu}$ nel punto $(x_0,y_0)$ è la derivata classica di:
		\begin{tcenter}
$t \to f((x_0,y_0)+t\cdot\overrightarrow{\nu})$
		\end{tcenter} 
cioè:
		\begin{tcenter}
$\mathbf{\Lim{f\to0} \frac{f((x_0,y_0)+h\cdot\overrightarrow{\nu})-f(x_0,y_0)}{h}}$
		\end{tcenter}
e si indica con $\frac{\partial f}{\partial \overrightarrow{\nu}}(x_0,y_0)$.
		\begin{descr}{1}
\item[Teorema] $\rightarrow$ Se $f$ è differenziabile, allora:
			\begin{tcenter}
$\frac{\partial f}{\partial\overrightarrow{\nu}}(x,y)=\nabla f(x,y)\scalare\overrightarrow{\nu}$
			\end{tcenter}
		\end{descr}
		
		
%TODO Inserire esempio finale lezione 5		
		
		

	\end{formulario}
\end{document}