\documentclass[8pt]{extarticle}
\usepackage[a4paper,headsep=10pt, top=30pt, bottom=30pt, left=30pt, right=30pt, footskip=10pt]{geometry}
\usepackage{listings}	
\usepackage{setspace}			%used for \begin{spacing}\end{spacing}-> vertical space setting
\usepackage{amsfonts} 			%used for \mathbb{} -> numerical sets
\usepackage{amsmath}			%used for \text{} inside math equations
\usepackage{enumitem}
\usepackage{bm}					%used for bold math expressions
\usepackage{calc}				%used for algebric operations inside \newenvironment
\usepackage{fancyhdr}			%used for head and foot settings
%\usepackage{draftwatermark}		%used for watermark
\usepackage{graphics}



%%watermark settings
%\SetWatermarkText{Vici Francesco}
%\SetWatermarkScale{2}
%\SetWatermarkAngle{60}
%%end watermark settings



\fancypagestyle{plain}
{
\fancyhead{}\fancyfoot{}
\fancyhead[C]{\vspace{0pt}   Sommario Calcolatori}
\fancyfoot[R]{Vici Francesco}
\fancyfoot[C]{\thepage}
}
\pagestyle{plain}



\newenvironment{formulario}
{
\setlength{\columnsep}{3em}
\twocolumn
\lstset{tabsize=3}
\begin{spacing}{1}
\begin{flushleft}
}{
\end{flushleft}
\end{spacing}
}



\newenvironment{tcenter}{
  \par
  \centering
  \setlength{\parskip}{0pt} % Rimuovi spaziatura verticale
  \noindent
}{
  \par
}



\newenvironment{Descr}
{
	\begin{description}[topsep=0pt,itemsep=0pt,partopsep=0pt, parsep=0pt]
}{
	\end{description}
}



\newenvironment{descr}[1]
{
\setlist[description,1]{leftmargin=2em + (2em * #1),labelindent=0em + (2em * #1)}
\begin{description}[topsep=0pt,itemsep=0pt,partopsep=0pt, parsep=0pt]
}{
\end{description}
}




\newenvironment{myParagraph}[1]
{
\begin{tcenter}
\textbf{#1}
\end{tcenter}
}{
\myRule
}



\newcommand{\myRule}{\rule{250pt}{0.1pt}}
\newcommand{\bo}[1]{\textbf{#1}}
\newcommand{\co}[1]{\textit{#1}}
\newcommand{\R}{\mathbb{R}}
\newcommand{\N}{\mathbb{N}}
\newcommand{\la}{\leftarrow}
\newcommand{\ra}{\rightarrow}
\newcommand{\sse}{\leftrightarrow}
\newcommand{\floor}[1]{\lfloor #1 \rfloor}





\begin{document}\begin{formulario}


	\begin{myParagraph}{Introduzione}
		\begin{Descr}
			\item[Modello Von Neumann] $\ra$ è un modello al alto livello che suddivide il sistema di elaborazione (calcolatore) in tre componenti:
			\begin{Descr}
				\item[Memoria]  $\ra$ ospita dati e programmi (istruzioni) i quali possono essere visti come la stessa cosa (in alcuni modelli, come Harvard, queste due parti della memoria vengono divise);
				 \item[CPU]  $\ra$ i dati presenti nella memoria vengono eseguiti dalla CPU (in seguito ad una compilazione) che legge la memoria e vi scrive i risultati delle operazioni effettuate. Si divide in due parti:
				\begin{Descr}
					\item[Unità Operativa (UO)]  $\ra$ esegue praticamente le operazioni sui dati;
					\item[Unità di Controllo (UC)]  $\ra$ riceve istruzioni che legge dalla memoria e comanda le funzioni che l'unità operativa deve svolgere. Inoltre legge le condizioni poste dall'unità operativa durante le operazioni.
				\end{Descr}
				Ci sono due modi di effettuare le operazioni:
		\begin{Descr}
			\item[Asincrono] $\ra$ difficile da gestire perché si rischia di cadere nell'instabilità. Infatti la scrittura/lettura delle operazioni tra UC e UO non ha un'organizzazione;
			\item[Sincrono] $\ra$ il funzionamento della CPU è gestito da un CLOCK (con onda quadra) che consente di scrivere/leggere istruzioni solo quando c'è altra tensione nel segnale dato dal clock. Ad ogni ciclo di controllo l'Uc passa delle operazioni alla UO e legge le condizioni emesse dalla stessa. Il periodo è $T_c=\frac{1}{f_c}$ (oggi le frequenze sono nell'ordine dei $\text{GHz}=10^9 \text{ Hz}$). 
		\end{Descr}	
				\item[Sistema I/O]  $\ra$ serve per trasportare dati dall'interno all'esterno e viceversa.
			\end{Descr}
		\end{Descr}
	\end{myParagraph}
	
	
\end{formulario}\end{document}